\chapter*{Introduction}

\addcontentsline{toc}{chapter}{Introduction}

With the rise of big data architectures, the need for creating large graphs has increased dramatically these last few years. And to process these large graphs, hardware limits (Memory, Computational power, ...) started to present a very large obstacle. Our personal computers have limited resources so we can only process very limited graphs.
\\
\\
With this paper, we came across this golden rule: how to create large graphs without exhausting the memory? This question could be answered by designing a couple of strategies for graph creation.
\\
\\
This report defines our approach to implement the different graph creation strategies and eventually our machine learning model. It is organized into 5 chapters:
\\
\\
In the first chapter, we will start by giving a brief history on graph theory, state its importance and then we will modelize our problem mathematically and state our hypothesis and different strategies.
\\
\\
In the second chapter, we will first detail the different paradigms and approaches we followed. Then we will present our system design and explain in details its different components.
\\
\\
In the third chapter, we will start by  mentioning the different tools and languages used. And then we will describe our progress and the different challenges we have met.
\\
\\
In the fourth chapter, we will analyze our data in order to create an efficient model later on.
\\
\\
In the fifth chapter, we describe machine learning methodology and its different steps. Present our machine learning model in addition to the results.
